\documentclass[11pt]{article}
\usepackage{amsfonts,amsmath}
\usepackage{latexsym}
\setlength{\oddsidemargin}{.0in}
\setlength{\evensidemargin}{.0in}
\setlength{\textwidth}{6.5in}
\setlength{\topmargin}{-0.4in}
\setlength{\textheight}{8.5in}


\begin{document}
\textbf{Keeyon Ebrahimi}\\
\textbf{N14193968}\\
\textbf{Assignment 5}\\
\\ \\ \\
\textbf{Exercise 9.1:}\\
\\
P(X = -1) = $0.2$\\
P(X = 2 ) = $0.5$\\
P(X = 6 ) = $0.3$ \\
\\
Expected Value = $(-1 * 0.2) + (2 * 0.5) + (6 * 0.3)$ = \textbf{$2.6$} 
\\\\
Mean = (-1 + 2 + 6) / 3 = $2\frac{1}{3}$
\\\\
Variance = {\huge $\frac{(-1 - \frac{1}{3}) ^ 2 + (2 - \frac{1}{3}) ^ 2 + (6 - \frac{1}{3}) ^ 2}{3} = \frac{110}{9}$ }
\\\\
Standard Deviation = {\large $\sqrt{Variace} = \sqrt{\frac{110}{9}} = 3.496 $ }
\newpage

\textbf{Exercise 9.2:}
\\ \\
\begin{enumerate}
\item[(a)] Marginal Distributions
\\
$P(X) = [ (0.12 + 0.08 + 0.10), (0.20 + 0.04 + 0.25), (0.08 + 0.10 + 0.03)]$\\
$P(Y) = [ (0.12 + 0.20 + 0.08), (0.08 + 0.04 + 0.10), (0.10 + 0.25 + 0.03)]$\\
\\
{\Large
\textbf{$P(X) = [0.3, 0.49, 0.21]$}\\
\textbf{$P(Y) = [0.4, 0.22, 0.38]$}\\
}
\\
\begin{center}

  \begin{tabular}{ r || c | c | c || c |}
	 & -1 & 1 & 2 & $P(X)$\\ \hline
	 0 & 0.12 & 0.08 & 0.10 & 0.3 \\ \hline   
    1 & 0.20 & 0.04 & 0.25 & 0.49\\ \hline
    3 & 0.08 & 0.10 & 0.03 & 0.21\\ \hline \hline
	$P(Y)$ & 0.4 & 0.22 & 0.38
  \end{tabular}
\end{center}
\item[(b)] $X$ and $Y$ Independent?  \textbf{Solution: No}
\\ \\
Lets label our original joint distribution as $F$. If we are dealing with something that is independent, we should get $F(X, Y) = P(X) * P(Y)$ for all $(x, y)$ in range, or each cell in the table.  This is because $P(A, B) = P(A) * P(B)$ with independent events. \\

\begin{center}
{\Large \textbf{$F(X, Y)$}}\\ 
  \begin{tabular}{ r || c | c | c || c |}
	 & -1 & 1 & 2 & $P(X)$\\ \hline
	 0 & 0.12 & 0.08 & 0.10 & 0.3 \\ \hline   
    1 & 0.20 & 0.04 & 0.25 & 0.49\\ \hline
    3 & 0.08 & 0.10 & 0.03 & 0.21\\ \hline \hline
	$P(Y)$ & 0.4 & 0.22 & 0.38
  \end{tabular}
\end{center}
\begin{center}
{\Large \textbf{$P(X) * P(Y)$}}\\ 
  \begin{tabular}{ r || c | c | c |}
	 & -1 & 1 & 2 \\ \hline
	 0 & 0.12 & 0.07 & 0.11\\ \hline   
    1 & 0.20 & 0.11 & 0.19 \\ \hline
    3 & 0.08 & 0.05 & 0.08 \\ \hline
    \hline
  \end{tabular}
\end{center}
As we can see, when we multiply the margins, we do not get the same the same as $F(X,Y)$, so because $F(X, Y) \neq P(X) * P(Y)$, we know that they are not Independent.
\item[(c)] Exp(X) and Exp(Y)\\
\\
{\Large Exp(X) = $(0 * 0.3) + (1 * 0.49) + (3 * 0.21) = 1.12$}\\
{\Large Exp(Y) = $(-1 * 0.4) + (1 * 0.22) + (2 * 0.38) = 0.58$}\\
\\
\item[(d)] Distribution of $X + Y$. \\ \\
We must first find all the possible values for $X + Y$.  $X$ can be $[0, 1, 3]$.  $Y$ can be $[-1, 1, 2]$\\
This means that the possible values for $X + Y$ are $[-1, 1, 2, 0, 3, 4, 5]$, thus
\\\\
$P(-1) = 0.12$\\
$P(0) = 0.20$\\
$P(1) = 0.08$ \\
$P(2) = 0.10 + 0.04 + 0.08 = 0.22$\\
$P(3) = 0.25$\\
$P(4) = 0.10$\\
$P(5) = 0.03$\\\
\item[(e)] $P(X | Y = 2)$ and $P(Y | X = 1)$
\\
\begin{itemize}
\item[i. ]\textbf{$P(X | Y = 2)$}
\\ \\
We know that $$P(X | Y = 2) = \frac{P(X , Y = 2)}{P(Y = 2)}$$ 
Now lets compute
$$P(Y = 2) = 0.38$$ \\
\begin{center}
\textbf{This means that}
\end{center}
{\Large
$$P(X = 0 | Y = 2) = \frac{0.1}{0.38} = 0.26315 = 26.32\%$$
$$P(X = 1 | Y = 2) = \frac{0.25}{0.38} = 0.65789 = 65.79\%$$
$$P(X = 3 | Y = 2) = \frac{0.03}{0.38} = 0.07894 = 7.895\%$$
}
\item[ii. ] $P(Y | X = 1)$ \\ \\
  We know that $$P(Y | X = 1) = \frac{P(X = 1, Y)}{P(X = 1)}$$ 
Now lets compute
$$P(X = 1) = 0.49$$\\
\begin{center}
\textbf{This means that} \\
\end{center}
{\Large
$$P(Y = -1 | X = 1) = \frac{0.20}{0.49} = 0.40816 = 40.82\%$$
$$P(Y = 1 | X = 1) = \frac{0.04}{0.49} = 0.08163 = 8.163\%$$
$$P(Y = 2 | X = 1) = \frac{0.25}{0.49} = 0.5102 = 51.02\%$$
}
\end{itemize}
\end{enumerate}
\newpage
\textbf{Exercise 9.3: }\\
\end{document}