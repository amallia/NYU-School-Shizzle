%%me=0 student solutions, me=1 - my solutions, me=2 - assignment
\def\me{0}
\def\num{10}  %homework number
\def\due{Wednesday, November 26}  %due date
\def\course{CSCI-GA.1170-001/002 Fundamental Algorithms} %course name
\def\name{**** INSERT YOUR NAME HERE ****}
%
\iffalse
INSTRUCTIONS: replace # by the homework number.
(if this is not ps#.tex, use the right file name)

  Clip out the ********* INSERT HERE ********* bits below and insert
appropriate TeX code.  Once you are done with your file, run

  ``latex ps#.tex''

from a UNIX prompt.  If your LaTeX code is clean, the latex will exit
back to a prompt.  To see intermediate results, type

  ``xdvi ps#.dvi'' (from UNIX prompt)
  ``yap ps#.dvi'' (if using MikTex in Windows)

after compilation. Once you are done, run

  ``dvips ps#.dvi''

which should print your file to the nearest printer.  There will be
residual files called ps#.log, ps#.aux, and ps#.dvi.  All these can be
deleted, but do not delete ps1.tex. To generate postscript file ps#.ps,
run

  ``dvips -o ps#.ps ps#.dvi''

I assume you know how to print .ps files (``lpr -Pprinter ps#.ps'')
\fi
%
\documentclass[11pt]{article}
\usepackage{amsfonts,amsmath}
\usepackage{latexsym}
\setlength{\oddsidemargin}{.0in}
\setlength{\evensidemargin}{.0in}
\setlength{\textwidth}{6.5in}
\setlength{\topmargin}{-0.4in}
\setlength{\textheight}{8.5in}

\newcommand{\handout}[5]{
   \renewcommand{\thepage}{#1, Page \arabic{page}}
   \noindent
   \begin{center}
   \framebox{
      \vbox{
    \hbox to 5.78in { {\bf \course} \hfill #2 }
       \vspace{4mm}
       \hbox to 5.78in { {\Large \hfill #5  \hfill} }
       \vspace{2mm}
       \hbox to 5.78in { {\it #3 \hfill #4} }
      }
   }
   \end{center}
   \vspace*{4mm}
}

\newcounter{pppp}
\newcommand{\prob}{\arabic{pppp}}  %problem number
\newcommand{\increase}{\addtocounter{pppp}{1}}  %problem number

%first argument desription, second number of points
\newcommand{\newproblem}[2]{
\ifnum\me=0
\ifnum\prob>0 \newpage \fi
\increase
\setcounter{page}{1}
\handout{\name, Homework \num, Problem \arabic{pppp}}{\today}{Name: \name}{Due:
\due}{Solutions to Problem \prob\ of Homework \num\ (#2)}
\else
\increase
\section*{Problem \num-\prob~(#1) \hfill {#2}}
\fi
}

%\newcommand{\newproblem}[2]{\increase
%\section*{Problem \num-\prob~(#1) \hfill {#2}}
%}

\def\squarebox#1{\hbox to #1{\hfill\vbox to #1{\vfill}}}
\def\qed{\hspace*{\fill}
        \vbox{\hrule\hbox{\vrule\squarebox{.667em}\vrule}\hrule}}
\newenvironment{solution}{\begin{trivlist}\item[]{\bf Solution:}}
                      {\qed \end{trivlist}}
\newenvironment{solsketch}{\begin{trivlist}\item[]{\bf Solution Sketch:}}
                      {\qed \end{trivlist}}
\newenvironment{code}{\begin{tabbing}
12345\=12345\=12345\=12345\=12345\=12345\=12345\=12345\= \kill }
{\end{tabbing}}

%\newcommand{\eqref}[1]{Equation~(\ref{eq:#1})}

\newcommand{\hint}[1]{({\bf Hint}: {#1})}
%Put more macros here, as needed.
\newcommand{\room}{\medskip\ni}
\newcommand{\brak}[1]{\langle #1 \rangle}
\newcommand{\bit}[1]{\{0,1\}^{#1}}
\newcommand{\zo}{\{0,1\}}
\newcommand{\C}{{\cal C}}

\newcommand{\nin}{\not\in}
\newcommand{\set}[1]{\{#1\}}
\renewcommand{\ni}{\noindent}
\renewcommand{\gets}{\leftarrow}
\renewcommand{\to}{\rightarrow}
\newcommand{\assign}{:=}

\newcommand{\AND}{\wedge}
\newcommand{\OR}{\vee}

\newcommand{\For}{\mbox{\bf For }}
\newcommand{\To}{\mbox{\bf to }}
\newcommand{\Do}{\mbox{\bf Do }}
\newcommand{\If}{\mbox{\bf If }}
\newcommand{\Then}{\mbox{\bf Then }}
\newcommand{\Else}{\mbox{\bf Else }}
\newcommand{\While}{\mbox{\bf While }}
\newcommand{\Repeat}{\mbox{\bf Repeat }}
\newcommand{\Until}{\mbox{\bf Until }}
\newcommand{\Return}{\mbox{\bf Return }}


\begin{document}

\ifnum\me=0
%\handout{PS\num}{\today}{Name: **** INSERT YOU NAME HERE ****}{Due:
%\due}{Solutions to Problem Set \num}
%
%I collaborated with *********** INSERT COLLABORATORS HERE (INDICATING
%SPECIFIC PROBLEMS) *************.
\fi
\ifnum\me=1
\handout{PS\num}{\today}{Name: Yevgeniy Dodis}{Due: \due}{Solution
{\em Sketches} to Problem Set \num}
\fi
\ifnum\me=2
\handout{PS\num}{\today}{Lecturer: Yevgeniy Dodis}{Due: \due}{Problem
Set \num}
\fi

\newproblem{Greedy Topological Sort}{12 (+4) points}

\begin{itemize}

\item[(a)] (3 points) Assume directed graph $G$ is acyclic. Show that $G$ has at least one
vertex $v$ having no outgoing edges.

\ifnum\me<2
\begin{solution}
***************** INSERT PROBLEM \prob a SOLUTION HERE ***************
\end{solution}
\fi

\item[(b)] (5 points) Consider the following greedy algorithm for topological
sort of a directed graph $G$: ``Find a vertex $v$ with no outgoing
edges. If no such $v$ exists, output `cyclic'. Else put $v$ as the
last vertex in the topological sort, remove $v$ from $G$ (by also
removing all incoming edges to $v$), and recurse on the remaining
graph $G'$ on $(n-1)$ vertices''. If this algorithm is correct, prove
it, else give a counter-example.

\ifnum\me<2
\begin{solution}
***************** INSERT PROBLEM \prob b SOLUTION HERE ***************
\end{solution}
\fi

\item[(c)] (4 (+4) points) It is easy to implement the above algorithm in time
$O(mn)$. Show how to implement it in time $O(n^2)$. For {\bf extra
credit}, do it in time $O(m+n)$.

\ifnum\me<2
\begin{solution}
***************** INSERT PROBLEM \prob c SOLUTION HERE ***************
\end{solution}
\fi

\end{itemize}

\newproblem{Minimizing Maximal Weight Edge}{6 points}

Recall, MST finds a spanning sub-tree $T$ of the original graph minimizing the sum of edge weights in $T$: $\sum_{e\in T} w(e)$. Consider a related problem MST$'$ which attempts to find a spanning sub-tree $T'$ of the original graph minimizing the maximum edge weight in $T'$: $\sum_{e\in T'} w(e)$.
Show that the solution $T$ to MST is also an optimal solution $T'$ to MST$'$, and vice versa.

\ifnum\me<2
\begin{solution}
***************** INSERT PROBLEM \prob ~ SOLUTION HERE ***************
\end{solution}
\fi

\newproblem{Same weights}{10 points}

\begin{itemize}

\item[(a)] (4 points) Assume that all edge weights of an undirected graph $G$ are equal to the same number $w$. Design the fastest algorithm you can to compute the MST of $G$. Argue the correctness of the algorithm and state its run-time. Is it faster than the standard $O(m + n\log n)$ run-time of Prim?

\ifnum\me<2
\begin{solution}
***************** INSERT PROBLEM \prob a SOLUTION HERE ***************
\end{solution}
\fi

\item[(b)] (6 points) Now assume the all the edge weights are equal to $w$, except for a single edge $e'=(u',v')$ whose weight is $w'$ (note, $w'$ might be either larger or smaller than $w$). Show how to modify your solution in part (a) to compute the MST of $G$. What is the running time of your algorithm and how does it compare to the run-time you obtained in part (a) (or standard Prim)?

\ifnum\me<2
\begin{solution}
***************** INSERT PROBLEM \prob b SOLUTION HERE ***************
\end{solution}
\fi

\end{itemize}


\newproblem{Small Weights}{16 points}

Assume all edge weights in $G$ are integers from $1$ to $w$.

\begin{itemize}

\item[(a)] (8 points) Show how to modify Prim's algorithm to achieve running time $O(m+nw)$. Hence, if $w=O(1)$, you get optimal  time $O(m+n)$.

\ifnum\me<2
\begin{solution}
***************** INSERT PROBLEM \prob a SOLUTION HERE ***************
\end{solution}
\fi

\item[(b)] (4 points) Now assume $w=n$, so that the previous solution in part (a) is no longer faster than standard. Show how to modify Kruscal's algorithm instead of Prim's, so that it now takes time $O(m + n\log n)$, instead of $O(m\log n)$.

\ifnum\me<2
\begin{solution}
***************** INSERT PROBLEM \prob b SOLUTION HERE ***************
\end{solution}
\fi

\item[(c)] (4 points) What is the largest $w$ for which you can still maintain the
$O(m + n\log n)$ run-time in part $b$? In particular, can you tolerate $w=n^2$? $w=n^3$?

\ifnum\me<2
\begin{solution}
***************** INSERT PROBLEM \prob c SOLUTION HERE ***************
\end{solution}
\fi
\end{itemize}


\end{document}
