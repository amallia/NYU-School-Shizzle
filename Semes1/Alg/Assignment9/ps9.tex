%%me=0 student solutions, me=1 - my solutions, me=2 - assignment
\def\me{0}
\def\num{9}  %homework number
\def\due{Wednesday, November 19}  %due date
\def\course{CSCI-GA.1170-001/002 Fundamental Algorithms} %course name
\def\name{Keeyon Ebrahimi}
%
\iffalse
INSTRUCTIONS: replace # by the homework number.
(if this is not ps#.tex, use the right file name)

  Clip out the ********* INSERT HERE ********* bits below and insert
appropriate TeX code.  Once you are done with your file, run

  ``latex ps#.tex''

from a UNIX prompt.  If your LaTeX code is clean, the latex will exit
back to a prompt.  To see intermediate results, type

  ``xdvi ps#.dvi'' (from UNIX prompt)
  ``yap ps#.dvi'' (if using MikTex in Windows)

after compilation. Once you are done, run

  ``dvips ps#.dvi''

which should print your file to the nearest printer.  There will be
residual files called ps#.log, ps#.aux, and ps#.dvi.  All these can be
deleted, but do not delete ps1.tex. To generate postscript file ps#.ps,
run

  ``dvips -o ps#.ps ps#.dvi''

I assume you know how to print .ps files (``lpr -Pprinter ps#.ps'')
\fi
%
\documentclass[11pt]{article}
\usepackage{amsfonts,amsmath}
\usepackage{latexsym}
\setlength{\oddsidemargin}{.0in}
\setlength{\evensidemargin}{.0in}
\setlength{\textwidth}{6.5in}
\setlength{\topmargin}{-0.4in}
\setlength{\textheight}{8.5in}

\newcommand{\handout}[5]{
   \renewcommand{\thepage}{#1, Page \arabic{page}}
   \noindent
   \begin{center}
   \framebox{
      \vbox{
    \hbox to 5.78in { {\bf \course} \hfill #2 }
       \vspace{4mm}
       \hbox to 5.78in { {\Large \hfill #5  \hfill} }
       \vspace{2mm}
       \hbox to 5.78in { {\it #3 \hfill #4} }
      }
   }
   \end{center}
   \vspace*{4mm}
}

\newcounter{pppp}
\newcommand{\prob}{\arabic{pppp}}  %problem number
\newcommand{\increase}{\addtocounter{pppp}{1}}  %problem number

%first argument desription, second number of points
\newcommand{\newproblem}[2]{
\ifnum\me=0
\ifnum\prob>0 \newpage \fi
\increase
\setcounter{page}{1}
\handout{\name, Homework \num, Problem \arabic{pppp}}{\today}{Name: \name}{Due:
\due}{Solutions to Problem \prob\ of Homework \num\ (#2)}
\else
\increase
\section*{Problem \num-\prob~(#1) \hfill {#2}}
\fi
}

%\newcommand{\newproblem}[2]{\increase
%\section*{Problem \num-\prob~(#1) \hfill {#2}}
%}

\def\squarebox#1{\hbox to #1{\hfill\vbox to #1{\vfill}}}
\def\qed{\hspace*{\fill}
        \vbox{\hrule\hbox{\vrule\squarebox{.667em}\vrule}\hrule}}
\newenvironment{solution}{\begin{trivlist}\item[]{\bf Solution:}}
                      {\qed \end{trivlist}}
\newenvironment{solsketch}{\begin{trivlist}\item[]{\bf Solution Sketch:}}
                      {\qed \end{trivlist}}
\newenvironment{code}{\begin{tabbing}
12345\=12345\=12345\=12345\=12345\=12345\=12345\=12345\= \kill }
{\end{tabbing}}

%\newcommand{\eqref}[1]{Equation~(\ref{eq:#1})}

\newcommand{\hint}[1]{({\bf Hint}: {#1})}
%Put more macros here, as needed.
\newcommand{\room}{\medskip\ni}
\newcommand{\brak}[1]{\langle #1 \rangle}
\newcommand{\bit}[1]{\{0,1\}^{#1}}
\newcommand{\zo}{\{0,1\}}
\newcommand{\C}{{\cal C}}

\newcommand{\nin}{\not\in}
\newcommand{\set}[1]{\{#1\}}
\renewcommand{\ni}{\noindent}
\renewcommand{\gets}{\leftarrow}
\renewcommand{\to}{\rightarrow}
\newcommand{\assign}{:=}

\newcommand{\AND}{\wedge}
\newcommand{\OR}{\vee}

\newcommand{\For}{\mbox{\bf For }}
\newcommand{\To}{\mbox{\bf to }}
\newcommand{\Do}{\mbox{\bf Do }}
\newcommand{\If}{\mbox{\bf If }}
\newcommand{\Then}{\mbox{\bf Then }}
\newcommand{\Else}{\mbox{\bf Else }}
\newcommand{\While}{\mbox{\bf While }}
\newcommand{\Repeat}{\mbox{\bf Repeat }}
\newcommand{\Until}{\mbox{\bf Until }}
\newcommand{\Return}{\mbox{\bf Return }}


\begin{document}

\ifnum\me=0
%\handout{PS\num}{\today}{Name: **** INSERT YOU NAME HERE ****}{Due:
%\due}{Solutions to Problem Set \num}
%
%I collaborated with *********** INSERT COLLABORATORS HERE (INDICATING
%SPECIFIC PROBLEMS) *************.
\fi
\ifnum\me=1
\handout{PS\num}{\today}{Name: Yevgeniy Dodis}{Due: \due}{Solution
{\em Sketches} to Problem Set \num}
\fi
\ifnum\me=2
\handout{PS\num}{\today}{Lecturer: Yevgeniy Dodis}{Due: \due}{Problem
Set \num}
\fi

\newproblem{Party Propaganda}{18 Points}

You have an undirected graph $G = (V,E)$ and two special nodes $r,d\in
V$. At time $0$, node $r$ is republican, node $d$ is democratic, while
all the other nodes $v\nin \{r,d\}$ are initially ``undecided''. For
every $i=1,2,3,\ldots$, the following 2-stage ``conversion'' process
is performed at time time $i$. At the first stage, all republicans at
time $(i-1)$ look at all their neighboring nodes $v$ which are still
undecided, and convert those undecided nodes to become
republican. Similarly, at the second stage, all democratic nodes at
time $(i-1)$ look at all their neighboring nodes $v$ which are still
undecided by the end of the first stage above, and convert those
undecided nodes to become democratic. The process is repeated until no
new conversions can be made. For example, if $G$ is a $5$-cycle
$1,2,3,4,5$ where $r=1$, $d=5$, after time $1$ node $2$ becomes
republican and node $4$ becomes democratic, and after time $2$ the
last remaining node $3$ becomes republican (as republicans move
first). On the other hand, if the initial democratic node was $d=3$
instead, then already after step $1$ nodes $2$ and $5$ become
republican, and node $4$ becomes democratic, and no step $2$ is
needed.

Assume each node $v$ have a field $v.color$, where {\em red} means
republican, {\em blue} means democratic, and {\em white} means
undecided, so that, at time $0$, $r.color = red$, $d.color=blue$, and
all other nodes $v$ have $v.color= white$.

\begin{itemize}

\item[(a)] (5 points) Using two BFS calls, show how to properly fill the final
color of each node.

\ifnum\me<2
\begin{solution} \\
The way to solve with two BFS calls is you first run a BFS with the republican node as your source and for each node save the distance as $Blue\ Distance$.  You then run a BFS with the democrat node as your source, and have each node save the distance labeled $Red\ Distance$.
\\
\\
After that you compare every nodes $Red\ Distance$ with their $Blue\ Distance$.  If the $Red\ Distance < Blue\ Distance$, you then change the node's color to Red.  If the $Blue\ Distance \leq Red\ Distance$, then you change the node's color to Blue.
\\
The difference of having the republican color change when the distance is $\leq$ and the democrat color only change when the distance is $<$ is there because the republicans are being analyzed first as stated in the problem
\end{solution}
\fi

\item[(b)] (8 points) Show how speed up your procedure in part (a) by a factor of
$2$ (or more, depending on your implementation) by directly modifying
the BFS procedure given in the book. Namely, instead of computing
distances from the root node, you are computing the final colors of
each node, by essentially performing a {\em single}, appropriately
modified BFS traversal of $G$. Please write pseudocode, as it is {\em
very} similar to the standard BFS pseudocode, and is much easier to
grade. But briefly explain your code.

\ifnum\me<2
\begin{solution}\\
First a brief explanation, then pseudo code.  We first set the distance of the republican node the democrat node to 0.  We will run a modified BFS that will Enqueue all neighbors that are not the same color as itself.  We will call the current node $CN$ and the adjacent node $AN$.  When comparing all of its adjacent nodes, if the adjacent color is white, progress as normal.  If $AJ.color \neq White$, we run these checks\\ \\

If $$ CN.Color == Blue\ and\ AN.Color == Red$$ $$and$$ $$CN.Distance + 1 \leq AN.Distance$$ then we change $AN.Color$ to blue, and we enqueue $AN$. \\

If $$ CN.Color == Red\ and\ AN.Color == Blue$$ $$and$$ $$CN.Distance + 1 < AN.Distance$$ then we change $AN.Color$ to red, and we enqueue $AN$. \\
\\ \\
The reason the change from blue to red happens when distance is $\leq$ and the red to blue only happens with distance is $<$ is because red is analyzed first.  If the problem was changed to analyze blue first, we would change this. \\

This will give us a the correct color scheme with only one BFS search.  Here is the Pseudo Code.
\\ \\
\begin{verbatim}
// r = Initial Red
// b = Initial Blue
BFS(G, r, b)
{
    for each vertex u in G.V - {r, b}
    {
        u.color = white
        u.d = infinity
        u.pi = NIL
    }
    
    r.color = Red
    r.d = 0
    r.pi = NIL
    
    b.color = Red
    b.d = 0
    b.pi = NIL
    
    Q = EmptySet
    ENQUEUE(Q, r)    
}
ENQUEUE(Q, s)
{
    while Q != EmptySet
    {
        u = DEQUEUE(Q)
        for each v in G.adjacent[u]
        {
            if v.color == white
            {
                v.color = u.color
                v.d = u.d + 1
                v.pi = u
                ENQUEUE(Q, v)
            }
            
            if (u.color == Blue) and (v.color == Red)
            {
                if(u.d + 1 <= v.d)
                {
                    v.color = u.color
                    v.d = u.d + 1
                    v.pi = u
                    ENQUEUE(Q, v)
                }
            }
                
            if (u.color == Red) and (v.color == Blue)
            {
                if(u.d + 1 < v.d)
                {
                    v.color = u.color
                    v.d = u.d + 1
                    v.pi = u
                    ENQUEUE(Q, v)
                }
            }
        }
    }
}

\end{verbatim}
\end{solution}
\fi

\item[(c)] (5 points) Now assume that at time $0$ more than one node could be
republican or democratic. Namely, you are given as inputs some
disjoint subsets $R$ and $D$ of $V$, where nodes in $R$ are initially
republican and nodes in $D$ are initially democratic, but otherwise
the conversion process is the same. For concreteness, assume
$|R|=|D|=t$ for some $t\ge 1$ (so that parts (a) and (b) correspond to
$t=1$). Show how to generalize your solutions in parts (a) and (b) to
this more general setting. Given parts (a) and (b) took time
$O(|V|+|E|)$ (with different constants), how long would their
modifications take as a function of $t$, $|V|$, $|E|$? Which procedure
gives a faster solution?

\ifnum\me<2
\begin{solution}
\\
To generalize part(a), we would have to run a $BFS$ on each node in set $R$ and each node in set $D$. Then we would have to assign each nodes $Red.distance$ and $Blue.distance$ the the resulting minimum value that any of the $BFS$'s found.  Then we would compare each of the node's $Min(Red.distance)$ with $Min(Blue.distance)$ and determine color that way. \\\\
The running time of this algorithm would be $O(t(|V| + |E|))$ This is because we are running a $BFS$ $t$ times\\ \\
To generalize part (b), the only difference is instead of only setting the initial red and blue node's color and distance.  We will instead, initially go through the set D and set all distances to 0 and colors to B.  We will then Go through set R and set all colors to red and distances to 0.  \\ \\
We then start the $BFS$ giving it any single node from set $R$ or set $D$ and part (b) will solve this issue.  This will not change the running time of part (b), so this algorithm will give us a running time of $O(|V| + |E|)$ \\ \\
Part (b) gives a much faster solution because part (b) only runs through one $BFS$ while part (a) has to run a $BFS$ for each item in set $D$ and set $R$.
\end{solution}
\fi

\end{itemize}

\newproblem{Fast Route of a Knight}{5 Points}

Consider an $n\times n$ chessboard. In one move, a knight can go from
position $(i, j)$ to $(k, \ell)$ for $1 \leq i, j, k, \ell \leq n$ if either $|k-i| = 1$ and $|j - \ell| = 2$ or $|k - i| = 2$ and $|j -\ell| = 1$.
However, a knight is not allowed to go to a square that is already occupied by a piece of the same color. You are given a starting position $(s_x, s_y)$ and a desired final position $(f_x, f_y)$ of a black knight and an
 array $B[1 \ldots n][1 \ldots n]$ such that $B[i][j] = 1$ if $(i,j)$ is
occupied by a black piece, and $0$, otherwise. Give an $O(n^2)$ algorithm to find the smallest number of moves needed for the knight to reach from the starting position to the final position.

\ifnum\me<2
\begin{solution}
\\ To solve this, we will be running a modified $BFS$ to find the shortest distance to $(f_x, f_y)$.  As for the modifications, each node will now have a $(x, y)$ position.  Then our $BFS$ will be modified this way.

\begin{enumerate}
\item[1. ] In our modified chess $BFS$, when we check adjacent nodes, we will instead be checking all nodes that pass $|k-i| = 1$ and $|j - \ell| = 2$ or $|k - i| = 2$ and $|j -\ell| = 1$.  These are the only nodes that we can check as adjacent nodes.
\item[2. ] In our modified chess $BFS$, when we would originally check if the node color is white, instead, we just check if that nodes position of $B[x][y] == 1$.  If it is not equal to one, we then know that it is a possible spot for our knight to land, and we ENQUEUE this node
\end{enumerate}

This will have the desired $O(n^2)$ running time.  We know that the running time of a $BFS = O(|V| + |E|)$.  In this chess example, we know that we have a total of $n^2$ V's, and each V can have at max $8$ E's (Max amount of squares that pass $|k-i| = 1$ and $|j - \ell| = 2$ or $|k - i| = 2$ and $|j -\ell| = 1$ for an individual square).  Now if we ignore constants and replace the $O(|V| + |E|)$ with our newly discovered $n$ values, we will see our running time of $O(n^2)$.
\end{solution}
\fi

\newproblem{Connected Subgraphs} {6 points}

An undirected graph is said to be conncected if there is a path between any two vertices in the graph.
Given a connected undirected graph $G = (V, E)$, where $V = \{1, \ldots, n\}$, give an algorithm that runs in time $O(|V| + |E|)$ and finds a permutation $\pi: [n] \mapsto [n]$ such that the subgraph of $G$ induced by the vertices $\{\pi(1), \ldots, \pi(i)\}$ is connected for any $i \leq n$.
Which of BFS or DFS gives a better algorithm for this problem?

\ifnum\me<2
\begin{solution}
\\ This can be solved using a $BFS$ or a $DFS$. 
\\ Subnote: For each below method, we just pick any random node in $V$ as our source for each different type of search.  This works because we are dealing with a connected undirected graph\\
\begin{enumerate}
\item[$DFS:$] We run a $DFS$ using any node in the connected graph as our source.  Whenever we visit a node and change its color to gray, which is the first step of $DFS-Visit$, we then append that node onto our permutation subgraph.  This will make our permutation order nodes by their time when initially visited.  By definition, all nodes at $t + 1$ can be visited from nodes at time $t$, hence that is how the $DFS$ assigned those value, so we know our permutation is correct.
\item[$BFS:$]  Very similar to the previous algorithm, we will run a $BFS$ using any node in the connected graph as our source.  When we visit a node and change it's color to gray, we will then append that value to our permutation order.  This will then order our permutation by its distance to $\pi(1)$, which is the source.  Because of how $BFS$ works, we will know that later nodes in the permutation will have a route to earlier nodes in the permutation, because at worst case, they can traverse back to the source and then take the source's path to any other node in the permutation. 
\end{enumerate}

They both give permutations that fit the given restrictions needed, but the $DFS$ gives more pertinent information as well.  The $DFS$ approach will closer mimic the actual traversal from one node to the next.  In the $DFS$ approach, elements right next to each other in the permutation will actually be close to each other in the graph, and we cannot say the same for the $BFS$ approach.  In the $BFS$ permutation, $\pi(i)$ and $\pi(i+1)$ can have the same distance to the source node, but be on complete opposite sides of $G$, so even though they are neighbors in the permutation, they are not actual neighbors by any means in $G$.  The $DFS$ approach though will have permutation neighbors that will be visited at one time step after itself, making us know that they are actually close in $G$ as well.
\end{solution}
\fi

\newproblem{Dividing the Class into Sections}{8 points}

The class teacher of a kindergarten class wishes to divide the class of $n$ children into two sections. She knows that some students pairs of students are friends with each other, and she wants to try to
 split the two sections in such a way that in each section all students are friends of each other.
Can you help her find an efficient algorithm to form the two sections given as input $n$, and $m$ statements of the form `$i$ and $j$ are friends with each other'. What is the running time of your algorithm?

\hint{Assume that the first student goes into the first section. Which section should the students who are friends of the first student go to? Which section should those that are not his friends go to?
Try to carefully form a graph and use BFS to solve this problem.}

\ifnum\me<2
\begin{solution} \\
This solution will do the following.  If there is an existing correct partition, this algorithm will divide the class into 2 sections with each section containing kids all friends with each other.  If there is no viable solution, then this algorithm will tell us. \\ \\
Basically, we will use $m$ statements to make a matrix containing the grid, we will then pick an arbitrary student as the source and run a $BFS$ with that source, and insert all students at distance 1 into section 1.  We then use the same $G$, with the same visited/non-visited student information, and run a $BFS$ on $G$ and we will use a non-visited student as the source of this $BFS$, appending all distance 1 students to this new source into Section 2.  If we have any remaining students that are not in Section 1 or Section 2, we know that we were given inputs with no possible solution.  This will have a running time of $O(n + m)$.  Here are those steps explained with more detail and with running time analysis. \\ \\

\item[1. ] We first need to create our graph, and we do this with our $m$ statements.  We will be storing our graph in an $n x n$ matrix.  Just how usual matrices store graphs, we will have the matrix at position $[x,y]$ store $1$ if there is a connection and $0$ if there is no connection.  For each $m$ statement that comes in, we will place a $1$ in our $G$ matrix at position $[i, j]$ and also at $[j, i]$.  This step will have a running time of $O(m)$
\item[2. ]  We then will run a $BFS$ using student $1$ as source. (Which student we pick as the source doesn't actually matter).  Whenever we hit a node and set the distance to $1$, we then append this into our $Secton\ 1$ Set.  Once we hit distance $2$, we will stop our $BFS$, for all distance 2 nodes will not be friends with the original source, and can no longer be in our section.  We will store this first distance 2 node though for later use.\\ \\
This step has a running time of $O(n)$.  This is because for our source, we are checking connections with each other possible student and there are $n$ students.  (If source student number is $s$, this will be like checking our matrix all values from $1\ to n$ for $G[s,x]$) This checks $n$ possible adjacent students in $G$.  We will also be running one step into distance 2, grabbing the first student that we hit that is not connected to student $s$\\
\item[3. ]  We will now have a competed Section 1 set that fits the restrictions of having all students be friends.  Now we will use the student with distance 2 to find our next section. We will label this student as $sn$ With our same $G$ from before, we will be running another $BFS$ using this new student, $sn$, as our source.  \\ \\
We will be using the same $G$ as step 2, because we want to preserve which nodes have been visited and which ones have not been visited from step 2.  We will now mimic the exact steps as step 2, placing all elements with Distance 1 from $sn$ into Set Section 2.  This will also take a running time of $O(n)$, just like step 2 did and for the same reason as step 2.  The only difference here is that if we hit a student that has a distance of two from $sn$ that has not yet been visited from Step 2 as well, we know that we were given a set of $n$ children with no possible solutions for dividing the students into 2 sets with each student in each set being friends. \\ \\
\begin{itemize}
\item Running Time Analysis:  As stated above, Step 1 takes $O(m)$, Step 2 takes $O(n)$ and Step 3 takes $O(n)$.  This means that we will have a total run time of $O(m + 2n) = O(m + n)$
\item Running Time = $O(m + n)$
\end{itemize}

\end{solution}
\fi

\newproblem{Lonely Vertex}{8 points}

\begin{itemize}

\item[(a)] (4 points) Explain how a vertex $u$ of a directed graph can end up in
a depth-first tree containing only $u$, although $u$ has both incoming
and outgoing edges.

\ifnum\me<2
\begin{solution}
\\ If a node has had all of it's outgoing edges have been visited and completely finished and traversed down by another traversal of the $DFS$, meaning other nodes have hit our nodes outgoing edges and have fully traversed down them already. 
\\ \\ If there exists a node that has incoming and outgoing edges that also has a sister node, which has the same incoming and outgoing edges as our original node does.  Then depending on which way the $DFS$ traverses, one of these nodes with incoming and outgoing edges will be in a depth-first tree with only itself.  Here is an example:\\ \\
\textbf{Example: } \\
So lets say we have a source Node, which we will label as $A$, and this $A$ node points to Node $B$ and Node $C$. Now, both Node $B$ and Node $C$ have a directed edge pointing to Node $D$.  So both $B$ and $C$ have one incoming edge from node $A$, and they both have an outgoing edge to node $D$.  \\ \\
When the $DFS$ occurs, we start at source node $A$.  Then $A$ will traverse down to either $B$ or $C$.  Lets say for this example it traverses down node $B$.  Now from $B$, we traverse down to $C$.  Node $C$ has no outgoing edges, so we update finish time and pop back up to node $B$.  Node $B$ has no remaining non visited outgoing edges, so we update its finish time and pop back up to Node $A$.  Node $A$ now traverses to Node $B$.  \textbf{Node $B$ has already had all of its outgoing edges traversed to by a previous traversal, so it's start time will be one less than it's end time, making it be in its own depth-first tree}
\end{solution}
\fi

\item[(b)] (4 points) Assume $u$ is part of some directed cycle in $G$. Can $u$
still end up all by itself in the depth-first forest of $G$? Justify
your answer.\\ \hint{Recall the White Path Theorem.}

\ifnum\me<2
\begin{solution}
\\ You can still have $u$ end up by itself in the depth-first forest of $G$.  As stated in part (a), this occurs when all outgoing edges of $u$ have already been visited by a different traversal of $G$.  This can still happen if $u$ is in a directed cycle in $G$. \\
\\
Let us label $c$ as the first node of a cycle that the $BFS$ visits.  If $u$ is the last node of that cycle, meaning it has an outgoing edge connecting back to $c$ and creates our cycle, then $u$ will be in a depth-first forest by itself.  \\ \\
We are looking for a case where all outgoing edges have already been traversed.  If we are not at the last element in the cycle that points back to the source, then we have elements that can still be traversed down to, which means it will not be in it's own forest.  If $u$ is the last element of the cycle however, making the final cycle connection, then it's outgoing edge to the source will be visited for it was the source. This will make it's start time be exactly one less than its end time, making it be a part of its own Depth-First forest. 
\\ \\
\textbf{Example: }\\
Lets say we have a cycle that has node $A$, $B$, and $C$, where $A \to B \to C \to A$.  We have a cycle and $A$ is the first element of the cycle that we visit.  Once we get to node $C$, we have already visited all nodes that $C$ outgoes to, so $C$ will have and end time that is 1 more than its start time, making it be a part of its own depth-first forest.
\end{solution}
\fi

\end{itemize}


\end{document}
