%%me=0 student solutions, me=1 - my solutions, me=2 - assignment
\def\me{0}
\def\num{9}  %homework number
\def\due{Wednesday, November 19}  %due date
\def\course{CSCI-GA.1170-001/002 Fundamental Algorithms} %course name
\def\name{Keeyon Ebrahimi}
%
\iffalse
INSTRUCTIONS: replace # by the homework number.
(if this is not ps#.tex, use the right file name)

  Clip out the ********* INSERT HERE ********* bits below and insert
appropriate TeX code.  Once you are done with your file, run

  ``latex ps#.tex''

from a UNIX prompt.  If your LaTeX code is clean, the latex will exit
back to a prompt.  To see intermediate results, type

  ``xdvi ps#.dvi'' (from UNIX prompt)
  ``yap ps#.dvi'' (if using MikTex in Windows)

after compilation. Once you are done, run

  ``dvips ps#.dvi''

which should print your file to the nearest printer.  There will be
residual files called ps#.log, ps#.aux, and ps#.dvi.  All these can be
deleted, but do not delete ps1.tex. To generate postscript file ps#.ps,
run

  ``dvips -o ps#.ps ps#.dvi''

I assume you know how to print .ps files (``lpr -Pprinter ps#.ps'')
\fi
%
\documentclass[11pt]{article}
\usepackage{amsfonts,amsmath}
\usepackage{latexsym}
\setlength{\oddsidemargin}{.0in}
\setlength{\evensidemargin}{.0in}
\setlength{\textwidth}{6.5in}
\setlength{\topmargin}{-0.4in}
\setlength{\textheight}{8.5in}

\newcommand{\handout}[5]{
   \renewcommand{\thepage}{#1, Page \arabic{page}}
   \noindent
   \begin{center}
   \framebox{
      \vbox{
    \hbox to 5.78in { {\bf \course} \hfill #2 }
       \vspace{4mm}
       \hbox to 5.78in { {\Large \hfill #5  \hfill} }
       \vspace{2mm}
       \hbox to 5.78in { {\it #3 \hfill #4} }
      }
   }
   \end{center}
   \vspace*{4mm}
}

\newcounter{pppp}
\newcommand{\prob}{\arabic{pppp}}  %problem number
\newcommand{\increase}{\addtocounter{pppp}{1}}  %problem number

%first argument desription, second number of points
\newcommand{\newproblem}[2]{
\ifnum\me=0
\ifnum\prob>0 \newpage \fi
\increase
\setcounter{page}{1}
\handout{\name, Homework \num, Problem \arabic{pppp}}{\today}{Name: \name}{Due:
\due}{Solutions to Problem \prob\ of Homework \num\ (#2)}
\else
\increase
\section*{Problem \num-\prob~(#1) \hfill {#2}}
\fi
}

%\newcommand{\newproblem}[2]{\increase
%\section*{Problem \num-\prob~(#1) \hfill {#2}}
%}

\def\squarebox#1{\hbox to #1{\hfill\vbox to #1{\vfill}}}
\def\qed{\hspace*{\fill}
        \vbox{\hrule\hbox{\vrule\squarebox{.667em}\vrule}\hrule}}
\newenvironment{solution}{\begin{trivlist}\item[]{\bf Solution:}}
                      {\qed \end{trivlist}}
\newenvironment{solsketch}{\begin{trivlist}\item[]{\bf Solution Sketch:}}
                      {\qed \end{trivlist}}
\newenvironment{code}{\begin{tabbing}
12345\=12345\=12345\=12345\=12345\=12345\=12345\=12345\= \kill }
{\end{tabbing}}

%\newcommand{\eqref}[1]{Equation~(\ref{eq:#1})}

\newcommand{\hint}[1]{({\bf Hint}: {#1})}
%Put more macros here, as needed.
\newcommand{\room}{\medskip\ni}
\newcommand{\brak}[1]{\langle #1 \rangle}
\newcommand{\bit}[1]{\{0,1\}^{#1}}
\newcommand{\zo}{\{0,1\}}
\newcommand{\C}{{\cal C}}

\newcommand{\nin}{\not\in}
\newcommand{\set}[1]{\{#1\}}
\renewcommand{\ni}{\noindent}
\renewcommand{\gets}{\leftarrow}
\renewcommand{\to}{\rightarrow}
\newcommand{\assign}{:=}

\newcommand{\AND}{\wedge}
\newcommand{\OR}{\vee}

\newcommand{\For}{\mbox{\bf For }}
\newcommand{\To}{\mbox{\bf to }}
\newcommand{\Do}{\mbox{\bf Do }}
\newcommand{\If}{\mbox{\bf If }}
\newcommand{\Then}{\mbox{\bf Then }}
\newcommand{\Else}{\mbox{\bf Else }}
\newcommand{\While}{\mbox{\bf While }}
\newcommand{\Repeat}{\mbox{\bf Repeat }}
\newcommand{\Until}{\mbox{\bf Until }}
\newcommand{\Return}{\mbox{\bf Return }}


\begin{document}

\ifnum\me=0
%\handout{PS\num}{\today}{Name: **** INSERT YOU NAME HERE ****}{Due:
%\due}{Solutions to Problem Set \num}
%
%I collaborated with *********** INSERT COLLABORATORS HERE (INDICATING
%SPECIFIC PROBLEMS) *************.
\fi
\ifnum\me=1
\handout{PS\num}{\today}{Name: Yevgeniy Dodis}{Due: \due}{Solution
{\em Sketches} to Problem Set \num}
\fi
\ifnum\me=2
\handout{PS\num}{\today}{Lecturer: Yevgeniy Dodis}{Due: \due}{Problem
Set \num}
\fi

\newproblem{Party Propaganda}{18 Points}

You have an undirected graph $G = (V,E)$ and two special nodes $r,d\in
V$. At time $0$, node $r$ is republican, node $d$ is democratic, while
all the other nodes $v\nin \{r,d\}$ are initially ``undecided''. For
every $i=1,2,3,\ldots$, the following 2-stage ``conversion'' process
is performed at time time $i$. At the first stage, all republicans at
time $(i-1)$ look at all their neighboring nodes $v$ which are still
undecided, and convert those undecided nodes to become
republican. Similarly, at the second stage, all democratic nodes at
time $(i-1)$ look at all their neighboring nodes $v$ which are still
undecided by the end of the first stage above, and convert those
undecided nodes to become democratic. The process is repeated until no
new conversions can be made. For example, if $G$ is a $5$-cycle
$1,2,3,4,5$ where $r=1$, $d=5$, after time $1$ node $2$ becomes
republican and node $4$ becomes democratic, and after time $2$ the
last remaining node $3$ becomes republican (as republicans move
first). On the other hand, if the initial democratic node was $d=3$
instead, then already after step $1$ nodes $2$ and $5$ become
republican, and node $4$ becomes democratic, and no step $2$ is
needed.

Assume each node $v$ have a field $v.color$, where {\em red} means
republican, {\em blue} means democratic, and {\em white} means
undecided, so that, at time $0$, $r.color = red$, $d.color=blue$, and
all other nodes $v$ have $v.color= white$.

\begin{itemize}

\item[(a)] (5 points) Using two BFS calls, show how to properly fill the final
color of each node.

\ifnum\me<2
\begin{solution} \\
The way to solve with two BFS calls is you first run a BFS with the republican node as your source and for each node save the distance as $Blue\ Distance$.  You then run a BFS with the democrat node as your source, and have each node save the distance labeled $Red\ Distance$.
\\
\\
After that you compare every nodes $Red\ Distance$ with their $Blue\ Distance$.  If the $Red\ Distance < Blue\ Distance$, you then change the node's color to Red.  If the $Blue\ Distance \leq Red\ Distance$, then you change the node's color to Blue.
\\
The difference of having the republican color change when the distance is $\leq$ and the democrat color only change when the distance is $<$ is there because the republicans are being analyzed first as stated in the problem
\end{solution}
\fi

\item[(b)] (8 points) Show how speed up your procedure in part (a) by a factor of
$2$ (or more, depending on your implementation) by directly modifying
the BFS procedure given in the book. Namely, instead of computing
distances from the root node, you are computing the final colors of
each node, by essentially performing a {\em single}, appropriately
modified BFS traversal of $G$. Please write pseudocode, as it is {\em
very} similar to the standard BFS pseudocode, and is much easier to
grade. But briefly explain your code.

\ifnum\me<2
\begin{solution}\\
First a brief explanation, then pseudo code.  We first set the distance of the republican node the democrat node to 0.  We will run a modified BFS that will Enqueue all neighbors that are not the same color as itself.  We will call the current node $CN$ and the adjacent node $AN$.  When comparing all of its adjacent nodes, if the adjacent color is white, progress as normal.  If $AJ.color \neq White$, we run these checks\\ \\

If $$ CN.Color == Blue\ and\ AN.Color == Red$$ $$and$$ $$CN.Distance + 1 \leq AN.Distance$$ then we change $AN.Color$ to blue, and we enqueue $AN$. \\

If $$ CN.Color == Red\ and\ AN.Color == Blue$$ $$and$$ $$CN.Distance + 1 < AN.Distance$$ then we change $AN.Color$ to red, and we enqueue $AN$. \\
\\ \\
The reason the change from blue to red happens when distance is $\leq$ and the red to blue only happens with distance is $<$ is because red is analyzed first.  If the problem was changed to analyze blue first, we would change this. \\

This will give us a the correct color scheme with only one BFS search.  Here is the Pseudo Code.
\\ \\
\begin{verbatim}
// r = Initial Red
// b = Initial Blue
BFS(G, r, b)
{
    for each vertex u in G.V - {r, b}
    {
        u.color = white
        u.d = infinity
        u.pi = NIL
    }
    
    r.color = Red
    r.d = 0
    r.pi = NIL
    
    b.color = Red
    b.d = 0
    b.pi = NIL
    
    Q = EmptySet
    ENQUEUE(Q, r)    
}
ENQUEUE(Q, s)
{
    while Q != EmptySet
    {
        u = DEQUEUE(Q)
        for each v in G.adjacent[u]
        {
            if v.color == white
            {
                v.color = u.color
                v.d = u.d + 1
                v.pi = u
                ENQUEUE(Q, v)
            }
            
            if (u.color == Blue) and (v.color == Red)
            {
                if(u.d + 1 <= v.d)
                {
                    v.color = u.color
                    v.d = u.d + 1
                    v.pi = u
                    ENQUEUE(Q, v)
                }
            }
                
            if (u.color == Red) and (v.color == Blue)
            {
                if(u.d + 1 < v.d)
                {
                    v.color = u.color
                    v.d = u.d + 1
                    v.pi = u
                    ENQUEUE(Q, v)
                }
            }
        }
    }
}

\end{verbatim}
\end{solution}
\fi

\item[(c)] (5 points) Now assume that at time $0$ more than one node could be
republican or democratic. Namely, you are given as inputs some
disjoint subsets $R$ and $D$ of $V$, where nodes in $R$ are initially
republican and nodes in $D$ are initially democratic, but otherwise
the conversion process is the same. For concreteness, assume
$|R|=|D|=t$ for some $t\ge 1$ (so that parts (a) and (b) correspond to
$t=1$). Show how to generalize your solutions in parts (a) and (b) to
this more general setting. Given parts (a) and (b) took time
$O(|V|+|E|)$ (with different constants), how long would their
modifications take as a function of $t$, $|V|$, $|E|$? Which procedure
gives a faster solution?

\ifnum\me<2
\begin{solution}
\\
To generalize part(a), we would have to run a $BFS$ on each node in set $R$ and each node in set $D$. Then we would have to assign each nodes $Red.distance$ and $Blue.distance$ the the resulting minimum value that any of the $BFS$'s found.  Then we would compare each of the node's $Min(Red.distance)$ with $Min(Blue.distance)$ and determine color that way. \\\\
The running time of this algorithm would be $O(t(|V| + |E|))$ This is because we are running a $BFS$ $t$ times\\ \\
To generalize part (b), the only difference is instead of only setting the initial red and blue node's color and distance.  We will instead, initially go through the set D and set all distances to 0 and colors to B.  We will then Go through set R and set all colors to red and distances to 0.  \\ \\
We then start the $BFS$ giving it any single node from set $R$ or set $D$ and part (b) will solve this issue.  This will not change the running time of part (b), so this algorithm will give us a running time of $O(|V| + |E|)$ \\ \\
Part (b) gives a much faster solution because part (b) only runs through one $BFS$ while part (a) has to run a $BFS$ for each item in set $D$ and set $R$.
\end{solution}
\fi

\end{itemize}

\newproblem{Fast Route of a Knight}{5 Points}

Consider an $n\times n$ chessboard. In one move, a knight can go from
position $(i, j)$ to $(k, \ell)$ for $1 \leq i, j, k, \ell \leq n$ if either $|k-i| = 1$ and $|j - \ell| = 2$ or $|k - i| = 2$ and $|j -\ell| = 1$.
However, a knight is not allowed to go to a square that is already occupied by a piece of the same color. You are given a starting position $(s_x, s_y)$ and a desired final position $(f_x, f_y)$ of a black knight and an
 array $B[1 \ldots n][1 \ldots n]$ such that $B[i][j] = 1$ if $(i,j)$ is
occupied by a black piece, and $0$, otherwise. Give an $O(n^2)$ algorithm to find the smallest number of moves needed for the knight to reach from the starting position to the final position.

\ifnum\me<2
\begin{solution}
\\ To solve this, we will be running a modified $BFS$ to find the shortest distance to $(f_x, f_y)$.  As for the modifications, each node will now have a $(x, y)$ position.  Then our $BFS$ will be modified this way.

\begin{enumerate}
\item[1. ] In our modified chess $BFS$, when we check adjacent nodes, we will instead be checking all nodes that pass $|k-i| = 1$ and $|j - \ell| = 2$ or $|k - i| = 2$ and $|j -\ell| = 1$.  These are the only nodes that we can check as adjacent nodes.
\item[2. ] In our modified chess $BFS$, when we would originally check if the node color is white, instead, we just check if that nodes position of $B[x][y] == 1$.  If it is not equal to one, we then know that it is a possible spot for our knight to land, and we ENQUEUE this node
\end{enumerate}

This will have the desired $O(n^2)$ running time.  We know that the running time of a $BFS = O(|V| + |E|)$.  In this chess example, we know that we have a total of $n^2$ V's, and each V can have at max $8$ E's (Max amount of squares that pass $|k-i| = 1$ and $|j - \ell| = 2$ or $|k - i| = 2$ and $|j -\ell| = 1$ for an individual square).  Now if we ignore constants and replace the $O(|V| + |E|)$ with our newly discovered $n$ values, we will see our running time of $O(n^2)$.
\end{solution}
\fi

\newproblem{Connected Subgraphs} {6 points}

An undirected graph is said to be conncected if there is a path between any two vertices in the graph.
Given a connected undirected graph $G = (V, E)$, where $V = \{1, \ldots, n\}$, give an algorithm that runs in time $O(|V| + |E|)$ and finds a permutation $\pi: [n] \mapsto [n]$ such that the subgraph of $G$ induced by the vertices $\{\pi(1), \ldots, \pi(i)\}$ is connected for any $i \leq n$.
Which of BFS or DFS gives a better algorithm for this problem?

\ifnum\me<2
\begin{solution}
***************** INSERT PROBLEM \prob ~ SOLUTION HERE ***************
\end{solution}
\fi

\newproblem{Dividing the Class into Sections}{8 points}

The class teacher of a kindergarten class wishes to divide the class of $n$ children into two sections. She knows that some students pairs of students are friends with each other, and she wants to try to
 split the two sections in such a way that in each section all students are friends of each other.
Can you help her find an efficient algorithm to form the two sections given as input $n$, and $m$ statements of the form `$i$ and $j$ are friends with each other'. What is the running time of your algorithm?

\hint{Assume that the first student goes into the first section. Which section should the students who are friends of the first student go to? Which section should those that are not his friends go to?
Try to carefully form a graph and use BFS to solve this problem.}

\ifnum\me<2
\begin{solution}
***************** INSERT PROBLEM \prob ~ SOLUTION HERE ***************
\end{solution}
\fi

\newproblem{Lonely Vertex}{8 points}

\begin{itemize}

\item[(a)] (4 points) Explain how a vertex $u$ of a directed graph can end up in
a depth-first tree containing only $u$, although $u$ has both incoming
and outgoing edges.

\ifnum\me<2
\begin{solution}
***************** INSERT PROBLEM \prob a SOLUTION HERE ***************
\end{solution}
\fi

\item[(b)] (4 points) Assume $u$ is part of some directed cycle in $G$. Can $u$
still end up all by itself in the depth-first forest of $G$? Justify
your answer.\\ \hint{Recall the White Path Theorem.}

\ifnum\me<2
\begin{solution}
***************** INSERT PROBLEM \prob b SOLUTION HERE ***************
\end{solution}
\fi

\end{itemize}


\end{document}
