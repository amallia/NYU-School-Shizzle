\documentclass{article}
\usepackage[english]{babel}

\begin{document}
\textbf{Keeyon Ebrahimi}\\
\textbf{HW4}
\\ \\
\textbf{Sentence 1: } \\
Mr. Cotton said the terms of the emerging deal made it too risky and noted that a Republican president succeeding Mr. Obama could decide not to honor it.
\begin{verbatim}
Sentence:                               Mr. Cotton said the terms of the emerging deal made it too risky and noted that a Republican president succeeding Mr. Obama could decide not to honor it
Annotating Mr. Cotton  as constit cat=np
Annotating said  as constit cat=vgroup
Annotating the terms  as constit cat=np
Annotating deal  as constit cat=np
Annotating made  as constit cat=vgroup
Annotating it  as constit cat=np
Annotating noted  as constit cat=vgroup
Annotating a Republican president  as constit cat=np
Annotating Mr. Obama  as constit cat=np
Annotating could decide  as constit cat=vgroup
Annotating honor  as constit cat=vgroup-inf
Annotating it as constit cat=np

\end{verbatim}
In this sentence we have 11 correct phrases, 2 missed phrases and 1 Wrongly identified group.
\\ \\
It incorrectly broke up Republican president succeeding Mr. Obama as two different noun groups instead of one.
\\
\\
\textbf{Sentence 2: }\\
 The basics of the Apple Watch have been known since September, but now, a few weeks before the watch’s arrival in stores, Apple has finally revealed how much it will cost.
\begin{verbatim}
Sentence:                                The basics of the Apple Watch have been known since September, but now, a few weeks before the watch’s arrival in stores, Apple has finally revealed how much it will cost.
Annotating The basics  as constit cat=np
Annotating Apple Watch  as constit cat=np
Annotating have  as constit cat=vgroup
Annotating September as constit cat=np
Annotating a few weeks  as constit cat=np
Annotating the watch as constit cat=np
Annotating ’ as constit cat=np
Annotating s  as constit cat=vgroup
Annotating stores as constit cat=np
Annotating Apple  as constit cat=np
Annotating has  as constit cat=vgroup
Annotating it  as constit cat=np
Annotating will cost as constit cat=vgroup
\end{verbatim}

In this sentence we have correctly labeled 8 groups, missed 5 labels, and got 1 label incorrect.
\\
\\
\textbf{Sentence 3: }\\
The cheapest model is the Apple Watch Sport, the one tailored to athletes, which starts at \$349

\begin{verbatim}
Sentence:                               The cheapest model is the Apple Watch Sport, the one tailored to athletes, which starts at $349
Annotating The cheapest model  as constit cat=np
Annotating is  as constit cat=vgroup
Annotating Apple Watch Sport as constit cat=np
Annotating the one  as constit cat=np
Annotating athletes as constit cat=np
Annotating starts  as constit cat=vgroup
\end{verbatim}

This sentence has correctly labeled 6 different groups, and has missed 3 groups.
\\
\\
\textbf{Sentence 4:  }\\
The MacBook also includes a trackpad with sensors that can detect how hard a user is clicking.
\begin{verbatim}
Sentence:                               The MacBook also includes a trackpad with sensors that can detect how hard a user is clicking.
Annotating MacBook  as constit cat=np
Annotating includes  as constit cat=vgroup
Annotating a trackpad  as constit cat=np
Annotating sensors  as constit cat=np
Annotating can detect  as constit cat=vgroup
Annotating a user  as constit cat=np
Annotating is clicking as constit cat=vgroup-pass
\end{verbatim}

This sentence had 7 correct groups and has also missed 2 groups.
\\
\newpage
\textbf{Problem 2: }
Here is the new chunkPatterns.txt used in order to accept perfect tenses, quantifiers, and pre-nomial nouns.

\begin{verbatim}
//  pattern set for noun and verb groups

pattern set chunks;

//  patterns for noun groups

ng := 		det-pos? [constit cat=q]? [constit cat=adj]* [constit cat=n]+ |
		proper-noun |
		[constit cat=pro];

det-pos	    :=	[constit cat=det] |
		[constit cat=det]? [constit cat=n number=singular] "'s";

proper-noun :=	([token case=cap] | [undefinedCap])+;

when ng		add [constit cat=np];

//  patterns for active verb groups

vg :=		[constit cat=tv] |
		[constit cat=w] vg-inf |
		tv-vbe vg-ving;

vg-inf :=	[constit cat=v] |
		"be" vg-ving;

vg-ving :=	[constit cat=ving];

tv-vbe :=	"is" | "are" | "was" | "were" | "have" | "has";

vg-ven :=	[constit cat=ven] | tv-vbe?
		"been" vg-ving;

when vg		add [constit cat=vgroup];

//  patterns for passive verb groups

vg-pass :=	tv-vbe [constit cat=ven]+ vg-ving? |
		[constit cat=w] "be" [constit cat=ven];

when vg-pass	add [constit cat=vgroup-pass];

//  pattern for infinitival verb groups

to-vg :=	vg-inf;

when to-vg	add [constit cat=vgroup-inf];


\end{verbatim}
\newpage
\textbf{Problem 3. }
\\
1.  No improvement on sentence one.
\\
\\
2.  Improvement on sentence two.  We now correctly identify "have been known"

\begin{verbatim}
Sentence:                               The basics of the Apple Watch have been known since September, but now, a few weeks before the watch’s arrival in stores, Apple has finally revealed how much it will cost.
Annotating The basics  as constit cat=np
Annotating Apple Watch  as constit cat=np
Annotating have been known  as constit cat=vgroup-pass
Annotating September as constit cat=np
Annotating a few weeks  as constit cat=np
Annotating the watch’ as constit cat=np
Annotating s  as constit cat=vgroup
Annotating stores as constit cat=np
Annotating Apple  as constit cat=np
Annotating has  as constit cat=vgroup
Annotating it  as constit cat=np
Annotating will cost as constit cat=vgroup

\end{verbatim}
3.  No improvement on sentence 3.
\\ 
\\
4.  No improvement on sentence 4.
\newpage
\textbf{Problem 4.}
These were found with Google ngram viewer. \\ \\
\textbf{Example 1}:
\\ "Traditionally, legislation \textbf{ has rarely been passed } by a vote strictly along party lines."
\\
American Government and Politics Today: The Essentials 2008 Pg 267. 
\\ \\
\textbf{Example 2}: 
\\ "Respiratory arrest \textbf{has rarely been} seen in children and adolescents."
\\ \\
Advanced Therapy in Epilepsy pg 287.

\end{document}

