\documentclass[ruled]{article}
\usepackage{relsize}

\begin{document}
\textbf{Keeyon Ebrahimi}\\
\textbf{Social Networks}\\
\textbf{HW2}\\ \\ \\
\textbf{Problem 1}

\begin{itemize}
\item[Q1] 
\begin{verbatim}

1. Consider the set of 18 Web pages drawn in Figure 13.8, with
links forming a directed graph. Which nodes constitute the largest
strongly connected component (SCC) in this graph? Taking this as
the giant SCC, which nodes then belong to the sets IN and OUT
defined in Section 13.4? Which nodes belong to the tendrils of the
graph?


\end{verbatim}
\textbf{Solution: }\\
\begin{itemize}
\item[(a)] SCC Nodes:  {\Large $1$, $8$, $13$, $18$, $3$, $9$, $14$, $4$, $15$} \\ \\

\item[(b)]  IN Nodes:  {\Large $6$, $7$, $11$, $12$ } \\ \\

\item[(c)]  OUT Nodes:  {\Large $5$, $10$, $16$}
\\
\\
\end{itemize}
\end{itemize}

\textbf{Problem 2}
\begin{itemize}
\item[Q2]
\begin{verbatim}

4. Figure 14.21 depicts the links among 6 Web pages, and also a
proposed PageRank value for each one, expressed as a decimal
next to the node. Are these correct equilibrium values for the Basic
PageRank Update Rule? Give a brief (1-3 sentence) explanation for
your answer.

\end{verbatim}
These are not correct equilibrium values for the Basic PageRank Update Rule.  If they were, another update would leave the proposed PageRank values to stay the same.  After another update we get these values.
\\ \\
{\Large 
$A = 0.15$ \\
$B = 0.15$ \\
$C = 0.075$ \\
$D = 0.25$ \\
$E = 0.075$ \\
$F = 0.3$ \\
}
\\
As we can see, nodes $A$, $B$, and $F$ have all kept their original values, but nodes $C$, $D$, and $E$ have not.  Nodes $C$ and $E$ have decreased and node $D$ has increased.   This means that the proposed PageRank values are not in equilibrium.\\
\end{itemize}

\end{document}
